\section*{Considerations}
This book is meant mainly for the guitar players that would like to have a chords book that is updated periodically without the losage of formatting and any other related problems.\par
It is highly recommended to print the content of this booklet and put it in a ring notebook - when a new song is added you can print it out and add it to a desired location. In order to avoid inconsistencies in adding new songs to your collection, songs are not labeled with any numbers. You can, for example, order the songs alphabetically.\par

\section*{Features}
\begin{itemize}
\item The booklet is available to be printed at once, including all the songs till date. Then you can expand it by adding the newly released songs without the necessity to reprint the whole booklet.
\item The format of a single sheet is ISO A5 which supports the idea of  portability.
\item Each song uses at most 2 pages whilst the first page would always be an even number in such case: you haven't got any problem about reading it.
\item Write the song with alphabetic note names and then you can generate that song in both alphabetic and solfege note names without rewriting it.
\item Easy automatic transpositions of the chords.
\item You are welcome to add your favourite songs with a pull request.
\item The book is licensed by a Free Culture License.
\item There is support for writing songs.
\item There is support for generating the booklets.
\end{itemize}

\section*{How to have a copy}
See \href{https://github.com/PietroPrandini/GuitarHub}{https://github.com/PietroPrandini/GuitarHub}

\begin{center}
  \qrcode[hyperlink,height=1.5cm]{https://github.com/PietroPrandini/GuitarHub}
\end{center}
