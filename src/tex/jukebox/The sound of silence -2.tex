\beginsong{%Title
The sound of silence}[
by={Simon and Garfunkel} % Authors, composers, and other contributors
%,cr={} % Copyright information
%,li={} % Licensing information
%,sr={} % Related scripture references
%,index={} % An extra index entry for a line of lyrics
%,ititle={} % An extra index entry for a hidden title
]

%\capo{0}
\transpose{11} % Automatic transpositions from +0 to +12 semitones

%	\beginverse* % * not count the verse
%		{\nolyrics Intro: }
%	\endverse

	\beginverse\memorize % \memorize is used to set the chords you would like to use with ^ in the next verses
		%verse
		\[Dm]Hello darkness, my old \[C]friend
		I've come to talk with you \[Dm]again
		Because a \[F]vision soft\[B&]ly creep\[F]ing
		Left its seeds while I \[B&]was sleep\[F]ing
		And the \[B&]vision that was planted in my \[F]brain
		Still rem\[Dm]ain
		\[F] Within the \[C]sound of \[Dm]silence
	\endverse
	\beginverse
		^In restless dreams I walked ^alone
		Narrow streets of cobbled ^stone
		'Neath the ^halo of ^a street ^lamp
		I turned my collar to the ^cold and ^damp
		When my ^eyes were stabbed by the flash of a neon ^light
		That split the ^night
		^ And touched the ^sound of ^silence
	\endverse
	\beginverse
		^And in the naked light I ^saw
		Ten thousand people, maybe ^more
		People ^talking with^out speak^ing
		People hearing with^out listen^ing
		People writing ^songs that voices never ^share
		No one ^dare
		^ Disturb the ^sound of ^silence
	\endverse
	\beginverse
		^"Fools" said I, "You do not ^know
		Silence like a cancer ^grows
		Hear my ^words that I ^might teach ^you
		Take my arms that I ^might reach ^you"
		But my ^words like silent raindrops ^fell
		^And ech^oed
		In the ^wells of ^silence
	\endverse
	\beginverse
		^And the people bowed and ^prayed
		To the neon god they ^made
		And the ^sign flashed out ^its warn^ing
		In the words that it ^was form^ing
		And the sign said
		"The ^words of the prophets are written on the subway ^walls
		And tenement ^halls"
		And ^whispered in the ^sounds of ^silence 
	\endverse

%	\beginchorus
		%chorus
%	\endchorus

%	\textnote{} % Notes for both lyric and chorded songs
%	\musicnote{} % Notes visible only in chorded books (not visible in lyric mode)
%	\rep{n} % Repeat n times

%	Writing chords
%
% Alphabetic note names:     A      B      C      D      E      F      G
% Solfedge note names:       LA     SI     DO     RE     MI     FA     SOL
%
%	Compatible notation:
%
% Naturals:                  \[A]   \[B]   \[C]   \[D]   \[E]   \[F]   \[G]
% Flat (Bemolle):            \[A&]  \[B&]  \[C&]  \[D&]  \[E&]  \[F&]  \[G&]
% Sharp (Diesis):            \[A#]  \[B#]  \[C#]  \[D#]  \[E#]  \[F#]  \[G#]
% Minor:                     \[Am]  \[Bm]  \[Cm]  \[Dm]  \[Em]  \[Fm]  \[Gm]
% Flat and minor:            \[A&m] \[B&m] \[C&m] \[D&m] \[E&m] \[F&m] \[G&m]
% Sharp and minor:           \[A#m] \[B#m] \[C#m] \[D#m] \[E#m] \[F#m] \[G#m]

%	You can print the book with solfege note names
%	by uncomment a line of the GuitarChords.tex:
%	```
%	%  \notenamesin{A}{B}{C}{D}{E}{F}{G}
  \notenamesout{LA}{SI}{DO}{RE}{MI}{FA}{SOL}
 % Solfedge note names
%	```

\endsong
