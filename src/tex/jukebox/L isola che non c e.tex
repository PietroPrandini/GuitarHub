\renewcommand{\lyricfont}{\sffamily\small}
\renewcommand{\printchord}[1]{\rmfamily\bf#1\small}

\beginsong{%Title
L'isola che non c'è}[
by={E. Bennato} % Authors, composers, and other contributors
,cr={\centering{\href{https://github.com/PietroPrandini/GuitarHub}{https://github.com/PietroPrandini/GuitarHub} \href{http://creativecommons.org/licenses/by-sa/4.0/}{CC-BY-SA} \filemodprintdate{"tex/jukebox/L isola che non c e.tex"}}}, % Copyright information
%,sr={} % Related scripture references
%,index={} % An extra index entry for a line of lyrics
%,ititle={} % An extra index entry for a hidden title
]

%\capo{0}
\transpose{0} % Automatic transpositions from +0 to +12 semitones
%\meter{4}{4}

%	\ifchorded
%	\beginverse* % * not count the verse
%		{\nolyrics Intro: }
%	\endverse
%	\fi

	\beginverse\memorize % \memorize is used to set the chords you would like to use with ^ in the next verses
		%verse
		Seconda \[C]stella a destra questo è il cam\[G]mino
		e poi \[F]dritto, fino al mat\[C]tino
		poi la \[Am]strada la \[E7]trovi da \[F]te
		porta al\[C]l'isola \[G] che non \[C]c'è.
	\endverse

	\beginverse
		Forse ^questo ti sembrerà ^strano
		ma la ^ragione ti ha un po' preso la ^mano
		ed ora ^sei quasi con^vinto ^che
		non può ^esistere un'^isola che non ^c'è.
	\endverse

	\beginverse
		E a pen^sarci, che paz^zia
		è una ^favola, è solo fanta^sia
		e chi è ^saggio, chi è ^maturo lo ^sa
		non può ^esistere ^ nella real^tà.
	\endverse

	\beginchorus
		Son d'ac\[Am]cordo con \[E7]voi
		non e\[Am]siste una \[E7]terra
		dove \[F]non ci son \[C]santi né e\[G]roi
		e se \[Dm]non ci son \[G7]ladri
		se \[Dm]non c'è mai la \[G7]guerra
		forse è \[Dm]proprio \[G7]l'isola
		che non \[Dm]c'è, che non \[G7]c'è.
	\endchorus

	\beginverse
		E non ^è un'inven^zione
		e nean^che un gioco di pa^role
		se ci ^credi ti ^basta per^ché
		poi la ^strada la ^trovi da ^te.
	\endverse

	\ifchorded
	\beginverse* % * not count the verse
		{\nolyrics Strum: \[C] \[G] \[F] \[C] \[Am] \[E7] \[F] \[C] \[G] \[C]}
	\endverse
	\fi

	\beginchorus
		Son d'ac\[Am]cordo con \[E7]voi
		niente \[Am]ladri e gen\[E7]darmi
		ma che \[F]razza di \[C]isola \[G]è?
		Niente \[Dm]odio e vio\[G]lenza
		né sol\[Dm]dati né \[G]armi
		forse è \[Dm]proprio \[G]l'isola
		che non \[Dm]c'è, che non \[G]c'è.
	\endchorus

	\beginverse
		Seconda ^stella a destra questo è il cam^mino
		e poi ^dritto, fino al mat^tino
		non ti ^puoi sba^gliare per^chè,
		quella è ^l'isola ^ che non ^c'è. \[C7]
	\endverse

	\beginverse
		E ti \[F]prendono in \[G]giro
		se con\[C]tinui a cer\[F]carla
		ma non \[C]darti per \[G]vinto per\[C]ché \[C7]
		chi ci ha \[F]già rinun\[G]ciato
		e ti \[C]ride alle \[F]spalle
		forse è \[C]ancora più \[G]pazzo di \[F]te.
	\endverse

%	\ifchorded
%	\beginverse* % * not count the verse
%		{\nolyrics Strum: }
%	\endverse
%	\fi

%	\textnote{} % Notes for both lyric and chorded songs
%	\musicnote{} % Notes visible only in chorded books (not visible in lyric mode)
%	\rep{n} % Repeat n times
%	\lrep ... \rrep \rep{n} % margins of the repeat

%	Writing chords
%
% Alphabetic note names:     A      B      C      D      E      F      G
% Solfedge note names:       LA     SI     DO     RE     MI     FA     SOL
%
%	Compatible notation:
%
% Naturals:                  \[A]   \[B]   \[C]   \[D]   \[E]   \[F]   \[G]
% Flat (Bemolle):            \[A&]  \[B&]  \[C&]  \[D&]  \[E&]  \[F&]  \[G&]
% Sharp (Diesis):            \[A#]  \[B#]  \[C#]  \[D#]  \[E#]  \[F#]  \[G#]
% Minor:                     \[Am]  \[Bm]  \[Cm]  \[Dm]  \[Em]  \[Fm]  \[Gm]
% Flat and minor:            \[A&m] \[B&m] \[C&m] \[D&m] \[E&m] \[F&m] \[G&m]
% Sharp and minor:           \[A#m] \[B#m] \[C#m] \[D#m] \[E#m] \[F#m] \[G#m]

\endsong

\renewcommand{\lyricfont}{\sffamily}
\renewcommand{\printchord}[1]{\rmfamily\bf#1}
