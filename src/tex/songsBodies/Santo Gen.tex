%\meter{4}{4}

	\ifchorded
	\beginverse* % * not count the verse
		{\nolyrics Intro: \[G] \[Em] \[Bm] \[D]}
	\endverse
	\fi

	\beginverse*\memorize % \memorize is used to set the chords you would like to use with ^ in the next verses
		%verse
		\[G]Santo, \[Em]Santo, \[Bm]Santo,
		il Si\[C]gnore Dio dell'uni\[D]verso.
		I \[C]cie\[Bm]li e la \[C]ter\[Em]ra
		sono \[C]pieni \[A]della Tua \[D]gloria.
	\endverse

	\beginchorus
		%chorus
		O\[G]san\[D]na, O\[G]san\[D]na,
		O\[C]sanna nell'\[D]alto dei \[G]cieli.
		O\[G]san\[D]na, O\[Em]san\[D]na,
		O\[C]sanna nell'\[D]alto dei \[G]cieli.
	\endchorus

	\beginverse*
		Bene\[G]detto co\[C]lui che \[D]viene
		nel \[D7]nome del Si\[G]gnore.
	\endverse

	\beginchorus
		%chorus
		O\[G]san\[D]na, O\[G]san\[D]na,
		O\[C]sanna nell'\[D]alto dei \[G]cieli.
		O\[G]san\[D]na, O\[Em]san\[D]na,
		O\[C]sanna nell'\[D]alto dei \[G]cieli.
	\endchorus

%	\ifchorded
%	\beginverse* % * not count the verse
%		{\nolyrics Strum: }
%	\endverse
%	\fi

%	\textnote{} % Notes for both lyric and chorded songs
%	\musicnote{} % Notes visible only in chorded books (not visible in lyric mode)
%	\rep{n} % Repeat n times
%	\lrep ... \rrep \rep{n} % margins of the repeat

%	Writing chords
%
% Alphabetic note names:     A      B      C      D      E      F      G
% Solfedge note names:       LA     SI     DO     RE     MI     FA     SOL
%
%	Compatible notation:
%
% Naturals:                  \[A]   \[B]   \[C]   \[D]   \[E]   \[F]   \[G]
% Flat (Bemolle):            \[A&]  \[B&]  \[C&]  \[D&]  \[E&]  \[F&]  \[G&]
% Sharp (Diesis):            \[A#]  \[B#]  \[C#]  \[D#]  \[E#]  \[F#]  \[G#]
% Minor:                     \[Am]  \[Bm]  \[Cm]  \[Dm]  \[Em]  \[Fm]  \[Gm]
% Flat and minor:            \[A&m] \[B&m] \[C&m] \[D&m] \[E&m] \[F&m] \[G&m]
% Sharp and minor:           \[A#m] \[B#m] \[C#m] \[D#m] \[E#m] \[F#m] \[G#m]
