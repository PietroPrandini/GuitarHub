%%% Writing chords %%%%%%%%%%%%%%%%%%%%%%%%%%%%%%%%%%%%%%%%%%%%%%%%%%%%%%%%%%%%%
% Alphabetic note names:     A      B      C      D      E      F      G       %
% Solfedge note names:       LA     SI     DO     RE     MI     FA     SOL     %
%                                                                              %
% Compatible notation:                                                         %
%                                                                              %
% Naturals:                  \[A]   \[B]   \[C]   \[D]   \[E]   \[F]   \[G]    %
% Flat (Bemolle):            \[A&]  \[B&]  \[C&]  \[D&]  \[E&]  \[F&]  \[G&]   %
% Sharp (Diesis):            \[A#]  \[B#]  \[C#]  \[D#]  \[E#]  \[F#]  \[G#]   %
% Minor:                     \[Am]  \[Bm]  \[Cm]  \[Dm]  \[Em]  \[Fm]  \[Gm]   %
% Flat and minor:            \[A&m] \[B&m] \[C&m] \[D&m] \[E&m] \[F&m] \[G&m]  %
% Sharp and minor:           \[A#m] \[B#m] \[C#m] \[D#m] \[E#m] \[F#m] \[G#m]  %
%                                                                              %
%%% Notes %%%%%%%%%%%%%%%%%%%%%%%%%%%%%%%%%%%%%%%%%%%%%%%%%%%%%%%%%%%%%%%%%%%%%%
% \textnote{}  % Notes for both lyric and chorded songs                        %
% \musicnote{} % Notes visible only in chorded books                           %
%                (not visible in lyric mode)                                   %
%                                                                              %
%%% Symbols %%%%%%%%%%%%%%%%%%%%%%%%%%%%%%%%%%%%%%%%%%%%%%%%%%%%%%%%%%%%%%%%%%%%
% \rep{n}                 % Repeat n times                                     %
% \lrep ... \rrep \rep{n} % Margins of the repeat                              %
% |                       % Measure bar                                        %
%                                                                              %
%%% Other Syntax %%%%%%%%%%%%%%%%%%%%%%%%%%%%%%%%%%%%%%%%%%%%%%%%%%%%%%%%%%%%%%%
% \beginverse ... \endverse   % Used for delimiters of a numbered verse        %
% *                           % Used after \beginverse for making an           %
%                             %   unnumbered verse                             %
% \memorize                   % Used after \beginverse for using chords        %
%                             %   with the symbol ^ in the next verses         %
% \beginchorus ... \endchorus % Used for delimiters of a chorus                %
% \ifchorded ... \fi          % Conditional sentences: only if chorded books   %
% \iflyric ... \fi           % Conditional sentences: only if lyrics book     %
% \nolyrics                   % For parts with no lyrics                       %
%                                                                              %
%%%%%%%%%%%%%%%%%%%%%%%%%%%%%%%%%%%%%%%%%%%%%%%%%%%%%%%%%%%%%%%%%%%%%%%%%%%%%%%%

% Start of the song body

%\meter{4}{4}

%\ifchorded
%	\beginverse*
%		{\nolyrics Intro: }
%	\endverse
%\fi

\beginverse\memorize
	\[G]Pane di vita \[C]sei,
	spez\[G]zato per tutti \[C]noi,
	chi ne \[C]man\[D]gia per \[G]sempre in \[C]Te vi\[D]vrà. \[D7]
	Ve\[G]niamo al Tuo santo al\[C]tare,
	\[G]mensa del Tuo a\[C]more,
	come \[C]pa\[D]ne \[G]vieni in \[C]mezzo a \[D4]noi. \[D]
\endverse

\beginchorus
	Il Tuo \[G]corpo ci sazie\[C]rà,
	il Tuo \[G]sangue ci salve\[C]rà,
	\[A7]perché Si\[G]gnor, Tu sei \[B]morto per a\[Em]more
	e ti \[A7]offri \[Am]og\[D]gi \[C]per \[D]noi.
	Il Tuo \[G]corpo ci sazie\[C]rà,
	il Tuo \[Em]sangue ci salve\[C]rà,
	\[A7]perché Si\[G]gnor, Tu sei \[B]morto per a\[Em]more
	e ti \[A7]offri \[Am]og\[D]gi per \[C]noi.  
\endchorus

\beginverse
	^Fonte di vita ^sei,
	^immensa cari^tà,
	il Tuo ^san^gue ci ^dona l'e^terni^tà. ^
	Ve^niamo al Tuo santo al^tare,
	^mensa del Tuo a^more,
	come ^vi^no ^vieni in ^mezzo a ^noi. ^
\endverse

%\ifchorded
%	\beginverse*
%		{\nolyrics Strum: }
%	\endverse
%\fi
