%%% Writing chords %%%%%%%%%%%%%%%%%%%%%%%%%%%%%%%%%%%%%%%%%%%%%%%%%%%%%%%%%%%%%
% Alphabetic note names:     A      B      C      D      E      F      G       %
% Solfedge note names:       LA     SI     DO     RE     MI     FA     SOL     %
%                                                                              %
% Compatible notation:                                                         %
%                                                                              %
% Naturals:                  \[A]   \[B]   \[C]   \[D]   \[E]   \[F]   \[G]    %
% Flat (Bemolle):            \[A&]  \[B&]  \[C&]  \[D&]  \[E&]  \[F&]  \[G&]   %
% Sharp (Diesis):            \[A#]  \[B#]  \[C#]  \[D#]  \[E#]  \[F#]  \[G#]   %
% Minor:                     \[Am]  \[Bm]  \[Cm]  \[Dm]  \[Em]  \[Fm]  \[Gm]   %
% Flat and minor:            \[A&m] \[B&m] \[C&m] \[D&m] \[E&m] \[F&m] \[G&m]  %
% Sharp and minor:           \[A#m] \[B#m] \[C#m] \[D#m] \[E#m] \[F#m] \[G#m]  %
%                                                                              %
%%% Notes %%%%%%%%%%%%%%%%%%%%%%%%%%%%%%%%%%%%%%%%%%%%%%%%%%%%%%%%%%%%%%%%%%%%%%
% \textnote{}  % Notes for both lyric and chorded songs                        %
% \musicnote{} % Notes visible only in chorded books                           %
%                (not visible in lyric mode)                                   %
%                                                                              %
%%% Symbols %%%%%%%%%%%%%%%%%%%%%%%%%%%%%%%%%%%%%%%%%%%%%%%%%%%%%%%%%%%%%%%%%%%%
% \rep{n}                 % Repeat n times                                     %
% \lrep ... \rrep \rep{n} % Margins of the repeat                              %
% |                       % Measure bar                                        %
%                                                                              %
%%% Other Syntax %%%%%%%%%%%%%%%%%%%%%%%%%%%%%%%%%%%%%%%%%%%%%%%%%%%%%%%%%%%%%%%
% \beginverse ... \endverse   % Used for delimiters of a numbered verse        %
% *                           % Used after \beginverse for making an           %
%                             %   unnumbered verse                             %
% \memorize                   % Used after \beginverse for using chords        %
%                             %   with the symbol ^ in the next verses         %
% \beginchorus ... \endchorus % Used for delimiters of a chorus                %
% \ifchorded ... \fi          % Conditional sentences: only if chorded books   %
% \iflyric ... \fi           % Conditional sentences: only if lyrics book     %
% \nolyrics                   % For parts with no lyrics                       %
%                                                                              %
%%%%%%%%%%%%%%%%%%%%%%%%%%%%%%%%%%%%%%%%%%%%%%%%%%%%%%%%%%%%%%%%%%%%%%%%%%%%%%%%

% Start of the song body

%\meter{4}{4}

\ifchorded
	\beginverse*
		{\nolyrics Intro: 	\[D] \[G] \[D] \[G]
					\[D] \[G] \[A4] \[A]}
	\endverse
\fi

\beginverse\memorize
	Che \[D]gioia ci hai \[G]dato, Si\[D]gnore del \[G]cielo
	Si\[D]gnore del \[G]grande uni\[A4]ver\[A]so!
	Che \[D]gioia ci hai \[G]dato, ve\[D]stito di \[G]luce
	ve\[D]stito di \[A]gloria infi\[Bm]ni\[G]ta,
	ve\[D]stito di \[A]gloria infi\[G]ni\[D]ta! \[G] \[D] \[G]
\endverse

\beginverse
	Ve^derTi ri^sorto, ve^derTi Si^gnore,
	il ^cuore sta ^per impaz^zi^re!
	Tu ^sei ritor^nato, Tu ^sei qui tra ^noi
	e a^desso Ti a^vremo per ^sem^pre,
	e a^desso Ti a^vremo per ^sem^pre. ^ ^ ^
\endverse

\beginverse
	^ Chi cercate, ^donne, quag^giù,
	chi cercate, ^donne, quag^giù?
	Quello che era ^morto non è ^qu^i:
	è ri^sorto, sì! come a^veva detto ^anche a voi,
	^voi gridate a ^tutti che ^
	^è risorto ^Lui,
	a ^tutti che ^
	^è risorto ^Lui! ^ ^ ^ 
\endverse

\beginverse
	^ Tu hai vinto il ^mondo, Ge^sù,
	Tu hai vinto il ^mondo, Ge^sù,
	liberiamo ^la felici^t^à!
	E la ^morte, no, non ^esiste più, l’hai ^vinta Tu
	e ^hai salvato ^tutti noi, ^
	^uomini con ^Te,
	^tutti noi, ^
	^uomini con ^Te.
\endverse

%1 e 2 strofa in sincrono con 3 e 4 niente sol re sol re sol prima di 5 e niente a fine di quinta, solo re

\beginverse
	\[G]Uomini con Te, uomini con Te.
	Che \[D]gioia ci hai \[G]dato, Ti a\[D]vremo per \[G]sem\[D]pre. \[G] \[D] \[G] \[D]
\endverse

%\ifchorded
%	\beginverse*
%		{\nolyrics Strum: }
%	\endverse
%\fi
