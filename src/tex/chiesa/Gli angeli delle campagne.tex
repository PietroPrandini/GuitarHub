%	Uncomment the next two lines and the last twice
%	if the song is over 2 pages
%\renewcommand{\lyricfont}{\sffamily\small}
%\renewcommand{\printchord}[1]{\rmfamily\bf#1\small}

\beginsong{%Title
Gli angeli delle campagne}[
%by={} % Authors, composers, and other contributors
,cr={\centering{\href{https://github.com/PietroPrandini/GuitarHub}{https://github.com/PietroPrandini/GuitarHub} - \href{http://creativecommons.org/licenses/by-sa/4.0/}{CC-BY-SA} - \filemodprintdate{"tex/chiesa/Gli angeli delle campagne.tex"}}}, % Copyright information
%,li={} % Licensing information
%,sr={} % Related scripture references
%,index={} % An extra index entry for a line of lyrics
%,ititle={} % An extra index entry for a hidden title
]

%\capo{0}
\transpose{0} % Automatic transpositions from +0 to +12 semitones
\meter{4}{4}

	\ifchorded
	\beginverse* % * not count the verse
		{\nolyrics Intro: |\[F] |\[C] \[F]| \rep{2}}
	\endverse
	\fi

	\beginverse\memorize % \memorize is used to set the chords you would like to use with ^ in the next verses
		%verse
		\[F]Gli angeli del\[C]le cam\[F]pagne
		\[F]cantano l’inno "\[C]Gloria in \[F]ciel!"
		\[F]E l’eco del\[C]le mon\[F]tagne
		\[F]ripete il canto \[C]dei fe\[F]del:
	\endverse

	\beginchorus
		%chorus
		| \[F]Gl\[Dm] | \[Gm]\[C] | \[F]\[B&] | \[G]o\[C]ria | \[F]in \[B&]ex\[F]cel\[B&]sis | \[F]De\[C]o. |
		| \[F]Gl\[Dm] | \[Gm]\[C] | \[F]\[B&] | \[G]o\[C]ria | \[F]in \[B&]ex\[F]cel\[B&]sis | \[F]D\[C]e\[F]o. |
	\endchorus

	\ifchorded
	\beginverse* % * not count the verse
		{\nolyrics Strum: |\[F] |\[C] \[F]| \rep{2}}
	\endverse
	\fi

	\beginverse
		^O pastori ^che can^tate
		^dite il perché di ^tanto o^nor.
		^Qual Signore, ^qual pro^feta
		^merita questo ^gran splen^dor.
	\endverse

	\beginverse
		^Oggi è nato in ^una ^stalla
		^nella notturna o^scuri^tà.
		^Egli è il Verbo, ^s’è incar^nato
		^e venne in questa ^pove^rtà.
	\endverse

%	\textnote{} % Notes for both lyric and chorded songs
%	\musicnote{} % Notes visible only in chorded books (not visible in lyric mode)
%	\rep{n} % Repeat n times
%	\lrep ... \rrep \rep{n} % margins of the repeat

%	Writing chords
%
% Alphabetic note names:     A      B      C      D      E      F      G
% Solfedge note names:       LA     SI     DO     RE     MI     FA     SOL
%
%	Compatible notation:
%
% Naturals:                  \[A]   \[B]   \[C]   \[D]   \[E]   \[F]   \[G]
% Flat (Bemolle):            \[A&]  \[B&]  \[C&]  \[D&]  \[E&]  \[F&]  \[G&]
% Sharp (Diesis):            \[A#]  \[B#]  \[C#]  \[D#]  \[E#]  \[F#]  \[G#]
% Minor:                     \[Am]  \[Bm]  \[Cm]  \[Dm]  \[Em]  \[Fm]  \[Gm]
% Flat and minor:            \[A&m] \[B&m] \[C&m] \[D&m] \[E&m] \[F&m] \[G&m]
% Sharp and minor:           \[A#m] \[B#m] \[C#m] \[D#m] \[E#m] \[F#m] \[G#m]

\endsong

%\renewcommand{\lyricfont}{\sffamily}
%\renewcommand{\printchord}[1]{\rmfamily\bf#1}
