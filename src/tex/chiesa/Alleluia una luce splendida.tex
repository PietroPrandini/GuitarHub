%	Uncomment the next two lines and the last twice
%	if the song is over 2 pages
%\renewcommand{\lyricfont}{\sffamily\small}
%\renewcommand{\printchord}[1]{\rmfamily\bf#1\small}

\beginsong{%Title
Alleluia, una luce splendida}[
by={F. Buttazzo} % Authors, composers, and other contributors
,cr={\centering{\href{https://github.com/PietroPrandini/GuitarHub}{https://github.com/PietroPrandini/GuitarHub} - \href{http://creativecommons.org/licenses/by-sa/4.0/}{CC-BY-SA} - \filemodprintdate{"tex/chiesa/Alleluia una luce splendida.tex"}}}, % Copyright information
%,li={} % Licensing information
%,sr={} % Related scripture references
%,index={} % An extra index entry for a line of lyrics
%,ititle={} % An extra index entry for a hidden title
]

%\capo{0}
\transpose{0} % Automatic transpositions from +0 to +12 semitones
%\meter{4}{4}

	\ifchorded
	\beginverse* % * not count the verse
		{\nolyrics Intro: \[G] \[D] \[C] \[D]}
	\endverse
	\fi

	\beginchorus
		\[G]Alleluia, \[D]alleluia, \[C]allelu\[G]ia!
		\[Am]Una luce \[G]splendida \[C]illumina la \[G]ter\[D]ra.
		\[G]Alleluia, \[D]alleluia, \[C]allelu\[G]ia!
		\[Am]Dio si è fatto \[G]uomo, è ve\[C]nuto in mezzo a \[G]noi.
	\endchorus

	\beginchorus
		\textnote{Ending}
		Alle\[Am]luia, \[D]allelu\[G]ia! \[C] \[G]
	\endchorus

	\beginverse\memorize % \memorize is used to set the chords you would like to use with ^ in the next verses
		%verse
		È spun\[G]tato per il \[D]mondo un giorno \[C]san\[D]to,
		Ado\[G]riamo il si\[D]gnore, il dio bam\[C]bi\[F#]no.
		Ascol\[Em]tiamo la sua voce, acco\[Am]gliamo la sua \[B]pace.
		Dentro il \[C]cuore di ogni \[G]uomo il suo \[Am]amore reste\[D]rà.
	\endverse

	\beginverse
		È ve^nuta sulla ^terra la sua pa^ro^la,
		e per ^noi la sua sa^pienza si è fatta ^car^ne.
		Lui ci ^dona la sua luce, che ri^splende nella ^notte
		e ci ^guida nella ^vita cammi^nando insieme a ^noi.
	\endverse

%	\beginchorus
		%chorus
%	\endchorus

%	\ifchorded
%	\beginverse* % * not count the verse
%		{\nolyrics Strum: }
%	\endverse
%	\fi

%	\textnote{} % Notes for both lyric and chorded songs
%	\musicnote{} % Notes visible only in chorded books (not visible in lyric mode)
%	\rep{n} % Repeat n times
%	\lrep ... \rrep \rep{n} % margins of the repeat

%	Writing chords
%
% Alphabetic note names:     A      B      C      D      E      F      G
% Solfedge note names:       LA     SI     DO     RE     MI     FA     SOL
%
%	Compatible notation:
%
% Naturals:                  \[A]   \[B]   \[C]   \[D]   \[E]   \[F]   \[G]
% Flat (Bemolle):            \[A&]  \[B&]  \[C&]  \[D&]  \[E&]  \[F&]  \[G&]
% Sharp (Diesis):            \[A#]  \[B#]  \[C#]  \[D#]  \[E#]  \[F#]  \[G#]
% Minor:                     \[Am]  \[Bm]  \[Cm]  \[Dm]  \[Em]  \[Fm]  \[Gm]
% Flat and minor:            \[A&m] \[B&m] \[C&m] \[D&m] \[E&m] \[F&m] \[G&m]
% Sharp and minor:           \[A#m] \[B#m] \[C#m] \[D#m] \[E#m] \[F#m] \[G#m]

\endsong

%\renewcommand{\lyricfont}{\sffamily}
%\renewcommand{\printchord}[1]{\rmfamily\bf#1}
