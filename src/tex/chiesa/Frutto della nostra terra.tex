%	Uncomment the next two lines and the last twice
%	if the song is over 2 pages
%\renewcommand{\lyricfont}{\sffamily\small}
%\renewcommand{\printchord}[1]{\rmfamily\bf#1\small}

\beginsong{%Title
Frutto della nostra terra}[
%by={} % Authors, composers, and other contributors
,cr={\centering{\href{https://github.com/PietroPrandini/GuitarHub}{https://github.com/PietroPrandini/GuitarHub} - \href{http://creativecommons.org/licenses/by-sa/4.0/}{CC-BY-SA} - \filemodprintdate{"tex/chiesa/Frutto della nostra terra.tex"}}}, % Copyright information
%,sr={} % Related scripture references
%,index={} % An extra index entry for a line of lyrics
%,ititle={} % An extra index entry for a hidden title
]

%\capo{0}
\transpose{0} % Automatic transpositions from +0 to +12 semitones
%\meter{4}{4}

	\ifchorded
	\beginverse* % * not count the verse
		{\nolyrics Intro: \[G] \[D] \[C] \[D]}
	\endverse
	\fi

	\beginverse\memorize % \memorize is used to set the chords you would like to use with ^ in the next verses
		%verse
		\[G]Frutto della nostra \[C]terra, \[G]del lavoro di ogni \[D]uomo:
		\[Em]pane della nostra \[Bm]vita, cibo \[C]della quotidiani\[D]tà.
		\[G]Tu che lo prendevi un g\[C]iorno, \[G]lo spezzavi per i \[D]tuoi,
		\[Em]oggi vieni in questo \[Bm]pane, cibo \[C]vero dell’umani\[D]tà.
	\endverse

	\beginchorus
		%chorus
		E sarò \[G]pane, e sarò \[D]vino nella mia \[Em]vita,
		nelle Tue \[Bm]mani. Ti accoglie\[C]rò dentro di \[D]me,
		farò di \[Em]me un’offerta \[C]viva,
		un sacri\[Am]ficio  \[D] gradito a \[G]Te.
	\endchorus

	\beginverse
		^Frutto della nostra ^terra, ^del lavoro di ogni ^uomo:
		^vino delle nostre ^vigne, sulla ^mensa dei fratelli ^tuoi.
		^Tu che lo prendevi un ^giorno, ^lo bevevi con i ^tuoi,
		^oggi vieni in questo ^vino e ti ^doni per la vita ^mia.
	\endverse

	\beginchorus
		%chorus
		E sarò \[G]pane, e sarò \[D]vino nella mia \[Em]vita,
		nelle Tue \[Bm]mani. Ti accoglie\[C]rò dentro di \[D]me,
		farò di \[Em]me un’offerta \[C]viva,
		un sacri\[Am]ficio  \[D] gradito a \[Em]Te, \[C]
		un sacri\[Am]ficio  \[D] gradito a \[G]Te.
	\endchorus

%	\ifchorded
%	\beginverse* % * not count the verse
%		{\nolyrics Strum: }
%	\endverse
%	\fi

%	\textnote{} % Notes for both lyric and chorded songs
%	\musicnote{} % Notes visible only in chorded books (not visible in lyric mode)
%	\rep{n} % Repeat n times
%	\lrep ... \rrep \rep{n} % margins of the repeat

%	Writing chords
%
% Alphabetic note names:     A      B      C      D      E      F      G
% Solfedge note names:       LA     SI     DO     RE     MI     FA     SOL
%
%	Compatible notation:
%
% Naturals:                  \[A]   \[B]   \[C]   \[D]   \[E]   \[F]   \[G]
% Flat (Bemolle):            \[A&]  \[B&]  \[C&]  \[D&]  \[E&]  \[F&]  \[G&]
% Sharp (Diesis):            \[A#]  \[B#]  \[C#]  \[D#]  \[E#]  \[F#]  \[G#]
% Minor:                     \[Am]  \[Bm]  \[Cm]  \[Dm]  \[Em]  \[Fm]  \[Gm]
% Flat and minor:            \[A&m] \[B&m] \[C&m] \[D&m] \[E&m] \[F&m] \[G&m]
% Sharp and minor:           \[A#m] \[B#m] \[C#m] \[D#m] \[E#m] \[F#m] \[G#m]

\endsong

%\renewcommand{\lyricfont}{\sffamily}
%\renewcommand{\printchord}[1]{\rmfamily\bf#1}
