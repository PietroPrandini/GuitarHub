%	Uncomment the next two lines and the last twice
%	if the song is over 2 pages
%\renewcommand{\lyricfont}{\sffamily\small}
%\renewcommand{\printchord}[1]{\rmfamily\bf#1\small}

\beginsong{%Title
Dal tronco di Iesse}[
%by={} % Authors, composers, and other contributors
,cr={\centering{\href{https://github.com/PietroPrandini/GuitarHub}{https://github.com/PietroPrandini/GuitarHub} - \href{http://creativecommons.org/licenses/by-sa/4.0/}{CC-BY-SA} - \filemodprintdate{"tex/chiesa/Dal tronco di Iesse.tex"}}}, % Copyright information
%,li={} % Licensing information
%,sr={} % Related scripture references
%,index={} % An extra index entry for a line of lyrics
%,ititle={} % An extra index entry for a hidden title
]

%\capo{0}
\transpose{0} % Automatic transpositions from +0 to +12 semitones
\meter{6}{8}

	\ifchorded
	\beginverse* % * not count the verse
		{\nolyrics Intro: | \[B&] \[F] | \[Gm] \[F] | \[E&] | \[F4] \[F] |}
	\endverse
	\fi

	\beginchorus
		%chorus
		Dal \[B&]tronco di \[F]Iesse \[E&]germoglie\[B&]rà
		un \[E&]nuovo vir\[B&]gulto do\[F]ma\[D]ni;
		dal\[Gm]le sue ra\[F]dici \[E&]si eleve\[B&]rà
		un \[Cm]albero \[F]nuo\[B&]vo. \rep{2}
	\endchorus

	\beginverse\memorize % \memorize is used to set the chords you would like to use with ^ in the next verses
		%verse
		Su di \[Gm]lui scende\[Dm]rà lo \[E&]Spirito di \[B&]Dio,
		gli \[Cm]regale\[Gm]rà i \[E&]suoi ricchi \[F]doni:
		con\[Gm]siglio e sa\[Dm]pienza, \[E&]scienza e for\[B&]tezza,
		\[Cm]santo ti\[E&]more di \[F]Dio.
	\endverse

	\beginverse
		Non ^giudiche^rà ^per le appa^renze,
		non ^decide^rà per ^sentito ^dire;
		ai ^poveri ^poi ^darà con lar^ghezza,
		fa^rà giu^stizia agli op^pressi.
	\endverse

	\beginverse
		Ed il ^lupo e l'a^gnello in ^pace vi^vranno,
		sa^ranno a^mici la ^mucca e il ^leone,
		^ed un fan^ciullo ^li guide^rà,
		^pascole^ranno in^sieme.
	\endverse

%	\ifchorded
%	\beginverse* % * not count the verse
%		{\nolyrics Strum: }
%	\endverse
%	\fi

%	\textnote{} % Notes for both lyric and chorded songs
%	\musicnote{} % Notes visible only in chorded books (not visible in lyric mode)
%	\rep{n} % Repeat n times
%	\lrep ... \rrep \rep{n} % margins of the repeat

%	Writing chords
%
% Alphabetic note names:     A      B      C      D      E      F      G
% Solfedge note names:       LA     SI     DO     RE     MI     FA     SOL
%
%	Compatible notation:
%
% Naturals:                  \[A]   \[B]   \[C]   \[D]   \[E]   \[F]   \[G]
% Flat (Bemolle):            \[A&]  \[B&]  \[C&]  \[D&]  \[E&]  \[F&]  \[G&]
% Sharp (Diesis):            \[A#]  \[B#]  \[C#]  \[D#]  \[E#]  \[F#]  \[G#]
% Minor:                     \[Am]  \[Bm]  \[Cm]  \[Dm]  \[Em]  \[Fm]  \[Gm]
% Flat and minor:            \[A&m] \[B&m] \[C&m] \[D&m] \[E&m] \[F&m] \[G&m]
% Sharp and minor:           \[A#m] \[B#m] \[C#m] \[D#m] \[E#m] \[F#m] \[G#m]

\endsong

%\renewcommand{\lyricfont}{\sffamily}
%\renewcommand{\printchord}[1]{\rmfamily\bf#1}
