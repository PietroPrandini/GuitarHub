\beginsong{%Title
Segni del Tuo amore}[
%by={} % Authors, composers, and other contributors
%,cr={} % Copyright information
%,li={} % Licensing information
%,sr={} % Related scripture references
%,index={} % An extra index entry for a line of lyrics
%,ititle={} % An extra index entry for a hidden title
]

%\capo{0}
\transpose{0} % Automatic transpositions from +0 to +12 semitones

	\ifchorded
	\beginverse*
		{\nolyrics Intro: \[C]\[C]\[C]\[Dm]\[C]\[Dm]\[C]\[Dm]\[C]\[Dm] \rep{2}}
	\endverse
	\fi


	\beginverse\memorize % \memorize is used to set the chords you would like to use with ^ in the next verses
		%verse
		\[C]Mille e mille \[C]grani \[C]nelle spighe \[Dm]d’or\[C]o \[Dm]\[C]\[Dm]\[C]\[Dm]
		\[C]mandano fra\[C]granza e \[C]danno gioia al \[Dm]cuo\[C]re, \[Dm]\[C]\[Dm]\[C]\[Dm]
		\[C]quando, maci\[C]nati \[C]fanno un pane \[Dm]sol\[C]o: \[Dm]\[C]\[Dm]\[C]\[Dm]
		\[C]pane quoti\[C]diano, \[C]dono Tuo, Si\[Dm]gno\[C]re. \[Dm]\[C]\[Dm]\[C]\[C]
	\endverse

	\beginchorus
		%chorus
		\[G]Ecco il pane e il vino, segni del Tuo a\[F]mo\[C]re.
		\[G]Ecco questa offerta, accoglila Si\[F]gno\[C]re:
		\[F]Tu di mille e mille \[G]cuori fai un cuore \[C]solo,
		un corpo solo in \[G]Te
		e il \[F]Figlio Tuo verrà, \[G]vivrà
		ancora in mezzo a noi.
	\endchorus

	\ifchorded
	\beginverse*
		{\nolyrics Strum: \[C]\[C]\[C]\[Dm]\[C]\[Dm]\[C]\[Dm]\[C]\[Dm]}
	\endverse
	\fi

	\beginverse
		^Mille grappo^li ma^turi sotto il ^so^le, ^^^^^
		^festa della ^terra, ^donano vi^go^re, ^^^^^
		^quando da ogni ^perla ^stilla il vino n^uo^vo: ^^^^^
		^vino della g^ioia, ^dono Tuo, Si^gno^re. ^^^^^
	\endverse

	\beginchorus
		%chorus
		\[G]Ecco il pane e il vino, segni del Tuo a\[F]mo\[C]re.
		\[G]Ecco questa offerta, accoglila Si\[F]gno\[C]re:
		\[F]Tu di mille e mille \[G]cuori fai un cuore \[C]solo,
		un corpo solo in \[G]Te
		e il \[F]Figlio Tuo verrà, \[G]vivrà
		ancora in mezzo a noi. \[F]\[C]
	\endchorus

	\beginchorus
		%chorus
		\[G]Ecco il pane e il vino, segni del Tuo a\[F]mo\[C]re.
		\[G]Ecco questa offerta, accoglila Si\[F]gno\[C]re:
		\[F]Tu di mille e mille \[G]cuori fai un cuore \[C]solo,
		un corpo solo in \[G]Te
		e il \[F]Figlio Tuo verrà, \[G]vivrà
		ancora in mezzo a noi.
	\endchorus

	\ifchorded
	\beginverse*
		{\nolyrics Strum: \[C]\[C]\[C]\[Dm]\[C]\[Dm]\[C]\[Dm]\[C]\[C]}
	\endverse
	\fi

%	\textnote{} % Notes for both lyric and chorded songs
%	\musicnote{} % Notes visible only in chorded books (not visible in lyric mode)
%	\rep{n} % Repeat n times

%	Writing chords
%
% Alphabetic note names:     A      B      C      D      E      F      G
% Solfedge note names:       LA     SI     DO     RE     MI     FA     SOL
%
%	Compatible notation:
%
% Naturals:                  \[A]   \[B]   \[C]   \[D]   \[E]   \[F]   \[G]
% Flat (Bemolle):            \[A&]  \[B&]  \[C&]  \[D&]  \[E&]  \[F&]  \[G&]
% Sharp (Diesis):            \[A#]  \[B#]  \[C#]  \[D#]  \[E#]  \[F#]  \[G#]
% Minor:                     \[Am]  \[Bm]  \[Cm]  \[Dm]  \[Em]  \[Fm]  \[Gm]
% Flat and minor:            \[A&m] \[B&m] \[C&m] \[D&m] \[E&m] \[F&m] \[G&m]
% Sharp and minor:           \[A#m] \[B#m] \[C#m] \[D#m] \[E#m] \[F#m] \[G#m]

%	You can print the book with solfege note names
%	by uncomment a line of the GuitarChords.tex:
%	```
%	%  \notenamesin{A}{B}{C}{D}{E}{F}{G}
  \notenamesout{LA}{SI}{DO}{RE}{MI}{FA}{SOL}
 % Solfedge note names
%	```

\endsong
