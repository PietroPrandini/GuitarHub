\beginsong{%Title
Ti loder\`o, Ti adorer\`o, Ti canter\`o}[
by={D. Basta - R. Baldi} % Authors, composers, and other contributors
,cr={\centering{\href{https://github.com/PietroPrandini/GuitarHub}{https://github.com/PietroPrandini/GuitarHub} \href{http://creativecommons.org/licenses/by-sa/4.0/}{CC-BY-SA} \filemodprintdate{"tex/chiesa/Ti lodero, Ti adorero, Ti cantero.tex"}}}, % Copyright information
%,sr={} % Related scripture references
%,index={} % An extra index entry for a line of lyrics
%,ititle={} % An extra index entry for a hidden title
]

%\capo{0}
\transpose{0} % Automatic transpositions from +0 to +12 semitones

	\ifchorded
	\beginverse* % * not count the verse
		{\nolyrics Intro: |  | \[Em] \[C] | \[G] \[D] | \[Em] \[C] | \[G] \[D] | \[D] | }
	\endverse
	\fi

	\beginverse\memorize % \memorize is used to set the chords you would like to use with ^ in the next verses
		%verse
		\[G]Vivi nel mio cuore, da \[D]quando Ti ho incontrato,
		\[C]sei con me, o Ges\[G]\`u.
		Ac\[G]cresci la mia fede, per\[D]ch\'e io possa amare
		\[C]come Te, o Ge\[G]s\`u.
		Per \[C]sempre io \[D]Ti di\[Em]r\`o il m\[G]io grazie,
		\[C] e in ete\[Am]rno cante\[D4]r\`o.\[D]
	\endverse

	\beginchorus
		%chorus
		Ti lode\[Em]r\`o, Ti adore\[C]r\`o, Ti cante\[G]r\`o che sei il mio \[D]Re.
		Ti lode\[Em]r\`o, Ti adore\[C]r\`o, benedi\[G]r\`o soltanto \[D]Te,
		chi \`e \[C]pari a \[D]Te Si\[Em]gnor, e\[G]terno amore \[C]sei,
		mio \[D]Salva\[G]tor ri\[C]sorto \[D]per \[Em]me.
		Ti adore\[C]r\`o, Ti cante\[G]r\`o che sei il mio \[D]Re,
		Ti lode\[Em]r\`o, Ti adore\[C]r\`o, benedi\[G]r\`o | soltanto \[D]Te. | \[D] |
	\endchorus

	\beginverse
		^Nasce in me, Signore, ^il canto della gioia,
		^grande sei, o Ge^s\`u.
		^Guidami nel mondo, se il ^buio \`e pi\`u profondo
		^splendi tu, o Ge^s\`u.
		Per ^sempre io ^Ti di^r\`o il m^io grazie,
		^ e in ete^rno cante^r\`o.^
	\endverse

	\beginchorus
		%chorus
		Ti lode\[Em]r\`o, Ti adore\[C]r\`o, Ti cante\[G]r\`o che sei il mio \[D]Re.
		Ti lode\[Em]r\`o, Ti adore\[C]r\`o, benedi\[G]r\`o soltanto \[D]Te,
		chi \`e \[C]pari a \[D]Te Si\[Em]gnor, e\[G]terno amore \[C]sei,
		mio \[D]Salva\[G]tor ri\[C]sorto \[D]per \[Em]me.
		Ti adore\[C]r\`o, Ti cante\[G]r\`o che sei il mio \[D]Re,
		Ti lode\[Em]r\`o, Ti adore\[C]r\`o, benedi\[G]r\`o | soltanto \[D]Te. | \[D]
	\endchorus

	\ifchorded
	\beginverse*
		{\nolyrics Strum: \[E] | \[F#m] \[D] | \[A] \[E] | \[F#m] \[D] |}
	\endverse
	\fi

	\transpose{2}
	\beginverse*
		| \[G] \[D] Ti lode\[Em]r\`o, | Ti adore\[C]r\`o,
		Ti cante\[G]r\`o che sei il mio \[D]Re.
		Ti lode\[Em]r\`o, Ti adore\[C]r\`o,
		benedi\[G]r\`o soltanto \[D]Te,
		\transpose{10}
		Ti lode\[A]r\`o,\[E] Ti adore\[F#m]r\`o, \[D]Ti cante\[A]r\`o.
		Ti lode\[E]r\`o, Ti adore\[F#m]r\`o, \[D]Ti cante\[A]r\`o.
	\endverse

%	\textnote{} % Notes for both lyric and chorded songs
%	\musicnote{} % Notes visible only in chorded books (not visible in lyric mode)
%	\rep{n} % Repeat n times

%	Writing chords
%
% Alphabetic noTe names:     A      B      C      D      E      F      G
% Solfedge noTe names:       LA     SI     DO     RE     MI     FA     SOL
%
%	Compatible notation:
%
% Naturals:                  \[A]   \[B]   \[C]   \[D]   \[E]   \[F]   \[G]
% Flat (Bemolle):            \[A&]  \[B&]  \[C&]  \[D&]  \[E&]  \[F&]  \[G&]
% Sharp (Diesis):            \[A#]  \[B#]  \[C#]  \[D#]  \[E#]  \[F#]  \[G#]
% Minor:                     \[Am]  \[Bm]  \[Cm]  \[Dm]  \[Em]  \[Fm]  \[Gm]
% Flat and minor:            \[A&m] \[B&m] \[C&m] \[D&m] \[E&m] \[F&m] \[G&m]
% Sharp and minor:           \[A#m] \[B#m] \[C#m] \[D#m] \[E#m] \[F#m] \[G#m]

%	You can print the book with solfege noTe names
%	by uncomment a line of the GuitarChords.tex:
%	```
%	%  \notenamesin{A}{B}{C}{D}{E}{F}{G}
  \notenamesout{LA}{SI}{DO}{RE}{MI}{FA}{SOL}
 % Solfedge noTe names
%	```

\endsong
