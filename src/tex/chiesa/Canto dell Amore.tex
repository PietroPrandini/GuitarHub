\beginsong{%Title
Canto dell'Amore}[
%by={} % Authors, composers, and other contributors
,cr={\centering{\href{https://github.com/PietroPrandini/GuitarHub}{https://github.com/PietroPrandini/GuitarHub} \href{http://creativecommons.org/licenses/by-sa/4.0/}{CC-BY-SA} \filemodprintdate{"tex/chiesa/Canto dell Amore.tex"}}}, % Copyright information
%,sr={} % Related scripture references
%,index={} % An extra index entry for a line of lyrics
%,ititle={} % An extra index entry for a hidden title
]

%\capo{0}
\transpose{0} % Automatic transpositions from +0 to +12 semitones

%	\beginverse* % * not count the verse
%		{\nolyrics Intro: }
%	\endverse

	\beginverse\memorize % \memorize is used to set the chords you would like to use with ^ in the next verses
		%verse
		Se \[E]dovrai attraversare il de\[C#m]serto
		non te\[A]mere Io sarò con \[E]te
		se dovrai camminare nel f\[C#m]uoco
		la sua f\[A]iamma non ti bruce\[E]rà
		segui\[B]rai la mia \[A]luce nella \[E]notte
		senti\[F#m]rai la mia \[B]forza nel camm\[C#m]ino
		io \[D]sono il tuo Dio, \[A] il Signore\[E]. \[C#m] \[A] \[E]
	\endverse

	\beginverse
		Sono ^Io che ti ho fatto e plas^mato
		ti ho ch^iamato per nome^
		Io da sempre ti ho conosc^iuto
		e ti ho ^dato il mio amore^
		perché ^tu sei pre^zioso ai miei oc^chi
		vali ^più del ^più grande dei te^sori
		Io ^sarò con te ^ dovunque and^rai. ^ ^ ^
	\endverse

	\ifchorded
	\beginverse*
		{\nolyrics Bridge \[B] \[A] \[E] \[D] \[A] \[B]}
	\endverse
	\fi

	\beginverse
		Non pen^sare alle cose di ^ieri
		cose n^uove fioriscono ^già
		aprirò nel deserto sent^ieri
		darò ^acqua nell'aridi^tà
		perché ^tu sei pre^zioso ai miei oc^chi
		vali ^più del ^più grande dei te^sori
		io ^sarò con te^ dovunque and^rai
		perché \[B]tu sei pre\[A]zioso ai miei oc\[E]chi
		vali \[F#m]più del \[B]più grande dei te\[C#m]sori
		io \[D]sarò con te \[A] dovunque and\[E]rai. \[C#m] \[A] \[E]
	\endverse

	\beginverse*
		Io ti sarò \[C#m] accanto \[A]sarò con \[E]te
		per tutto \[C#m] il tuo viaggio \[A]sarò con \[E]te. \rep{2}
	\endverse

%	\beginchorus
		%chorus
%	\endchorus

%	\textnote{} % Notes for both lyric and chorded songs
%	\musicnote{} % Notes visible only in chorded books (not visible in lyric mode)
%	\rep{n} % Repeat n times

%	Writing chords
%
% Alphabetic note names:     A      B      C      D      E      F      G
% Solfedge note names:       LA     SI     DO     RE     MI     FA     SOL
%
%	Compatible notation:
%
% Naturals:                  \[A]   \[B]   \[C]   \[D]   \[E]   \[F]   \[G]
% Flat (Bemolle):            \[A&]  \[B&]  \[C&]  \[D&]  \[E&]  \[F&]  \[G&]
% Sharp (Diesis):            \[A#]  \[B#]  \[C#]  \[D#]  \[E#]  \[F#]  \[G#]
% Minor:                     \[Am]  \[Bm]  \[Cm]  \[Dm]  \[Em]  \[Fm]  \[Gm]
% Flat and minor:            \[A&m] \[B&m] \[C&m] \[D&m] \[E&m] \[F&m] \[G&m]
% Sharp and minor:           \[A#m] \[B#m] \[C#m] \[D#m] \[E#m] \[F#m] \[G#m]

%	You can print the book with solfege note names
%	by uncomment a line of the GuitarChords.tex:
%	```
%	%  \notenamesin{A}{B}{C}{D}{E}{F}{G}
  \notenamesout{LA}{SI}{DO}{RE}{MI}{FA}{SOL}
 % Solfedge note names
%	```

\endsong
