%	Uncomment the next two lines and the last twice
%	if the song is over 2 pages
%\renewcommand{\lyricfont}{\sffamily\small}
%\renewcommand{\printchord}[1]{\rmfamily\bf#1\small}

\beginsong{%Title
Alleluia servire è}[
%by={} % Authors, composers, and other contributors
,cr={\centering{\href{https://github.com/PietroPrandini/GuitarHub}{https://github.com/PietroPrandini/GuitarHub} - \href{http://creativecommons.org/licenses/by-sa/4.0/}{CC-BY-SA} - \filemodprintdate{"tex/chiesa/Alleluia servire e.tex"}}}, % Copyright information
%,sr={} % Related scripture references
%,index={} % An extra index entry for a line of lyrics
%,ititle={} % An extra index entry for a hidden title
]

%\capo{0}
\transpose{0} % Automatic transpositions from +0 to +12 semitones
%\meter{4}{4}

%	\ifchorded
%	\beginverse* % * not count the verse
%		{\nolyrics Intro: }
%	\endverse
%	\fi

	\beginchorus
		%chorus
		\[F]Alle\[C]luia, ser\[B&]vire \[C]è,
		\[F]alle\[C]luia, ser\[B&]vire è \[C]gioia.
		\[F]Alle\[C]luia, ser\[B&]vire \[C]è,
		\[F]alle\[C]luia, è \[B&]stare con \[C]Te, Ge\[F]sù. \[C] \[B&] \[C]
	\endchorus

	\beginverse\memorize % \memorize is used to set the chords you would like to use with ^ in the next verses
		%verse
		\[F]Come l'amico più \[C]grande \[B&]tu guidi i miei \[F]passi,
		\[F]sei lungo la \[C]strada \[B&]e cammini con \[C]me.
	\endverse

	\beginverse
		^Tu sai accogliere ^tutti, ^Tu sai donare spe^ranza,
		^hai parole d'a^more, ^voglio restare con ^Te.
	\endverse

	\beginverse
		^Costruirò la mia ^casa ^su una roccia si^cura,
		^sei la perla pre^ziosa, ^sei un tesoro per ^me.
	\endverse

	\beginverse
		^Hai detto: voi siete la ^luce, ^siete granelli di ^sale
		^e non abbiate pa^ura: ^non vi abbandone^rò.
	\endverse

%	\ifchorded
%	\beginverse* % * not count the verse
%		{\nolyrics Strum: }
%	\endverse
%	\fi

%	\textnote{} % Notes for both lyric and chorded songs
%	\musicnote{} % Notes visible only in chorded books (not visible in lyric mode)
%	\rep{n} % Repeat n times
%	\lrep ... \rrep \rep{n} % margins of the repeat

%	Writing chords
%
% Alphabetic note names:     A      B      C      D      E      F      G
% Solfedge note names:       LA     SI     DO     RE     MI     FA     SOL
%
%	Compatible notation:
%
% Naturals:                  \[A]   \[B]   \[C]   \[D]   \[E]   \[F]   \[G]
% Flat (Bemolle):            \[A&]  \[B&]  \[C&]  \[D&]  \[E&]  \[F&]  \[G&]
% Sharp (Diesis):            \[A#]  \[B#]  \[C#]  \[D#]  \[E#]  \[F#]  \[G#]
% Minor:                     \[Am]  \[Bm]  \[Cm]  \[Dm]  \[Em]  \[Fm]  \[Gm]
% Flat and minor:            \[A&m] \[B&m] \[C&m] \[D&m] \[E&m] \[F&m] \[G&m]
% Sharp and minor:           \[A#m] \[B#m] \[C#m] \[D#m] \[E#m] \[F#m] \[G#m]

\endsong

%\renewcommand{\lyricfont}{\sffamily}
%\renewcommand{\printchord}[1]{\rmfamily\bf#1}
