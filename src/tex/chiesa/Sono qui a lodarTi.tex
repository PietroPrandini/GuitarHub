\beginsong{%Title
Sono qui a lodarTi}[
%by={} % Authors, composers, and other contributors
,cr={\centering{\href{https://github.com/PietroPrandini/GuitarHub}{https://github.com/PietroPrandini/GuitarHub} - \href{http://creativecommons.org/licenses/by-sa/4.0/}{CC-BY-SA} - \filemodprintdate{"tex/chiesa/Sono qui a lodarTi.tex"}}}, % Copyright information
%,sr={} % Related scripture references
%,index={} % An extra index entry for a line of lyrics
%,ititle={} % An extra index entry for a hidden title
]

%\capo{0}
\transpose{0} % Automatic transpositions from +0 to +12 semitones
%\meter{4}{4}

%	\ifchorded
%	\beginverse* % * not count the verse
%		{\nolyrics Intro: }
%	\endverse
%	\fi

	\beginverse\memorize % \memorize is used to set the chords you would like to use with ^ in the next verses
		%verse
		\[E]Luce del \[B]mondo nel \[F#m]buio del cuore,
		\[E]vieni ed il\[B]lumi\[A]nami,
		\[E]Tu, mia \[B]sola spe\[F#m]ranza di vita,
		\[E]resta per \[B]sempre \[A]con me.
	\endverse

	\beginchorus
		%chorus
		Sono qui a lo\[E]darTi, qui per ado\[B]rarTi,
		qui per dirTi \[C#m]che Tu sei il mio \[A]Dio
		e solo Tu sei \[E]Santo, sei meravi\[B]glioso,
		degno e glo\[C#m]rioso sei per \[A]me. \[A]
	\endchorus

	\beginverse
		^Re della ^storia e ^Re nella gloria,
		^sei sceso in ^terra ^fra noi.
		^Con umil^tà il Tuo ^trono hai lasciato
		^per dimo^strarci il ^Tuo amor.
	\endverse

	\beginchorus
		%chorus
		Sono qui a lo\[E]darTi, qui per ado\[B]rarTi,
		qui per dirTi \[C#m]che Tu sei il mio \[A]Dio
		e solo Tu sei \[E]Santo, sei meravi\[B]glioso,
		degno e glo\[C#m]rioso sei per \[A]me.
	\endchorus

	\beginverse*
		Non \[B]so quan\[C#m]to è co\[A]stato a Te
		mo\[B]rire in \[C#m]croce \[A]lì per me. \rep{2}
	\endverse


%	\ifchorded
%	\beginverse* % * not count the verse
%		{\nolyrics Strum: }
%	\endverse
%	\fi

%	\textnote{} % Notes for both lyric and chorded songs
%	\musicnote{} % Notes visible only in chorded books (not visible in lyric mode)
%	\rep{n} % Repeat n times
%	\lrep ... \rrep \rep{n} % margins of the repeat

%	Writing chords
%
% Alphabetic note names:     A      B      C      D      E      F      G
% Solfedge note names:       LA     SI     DO     RE     MI     FA     SOL
%
%	Compatible notation:
%
% Naturals:                  \[A]   \[B]   \[C]   \[D]   \[E]   \[F]   \[G]
% Flat (Bemolle):            \[A&]  \[B&]  \[C&]  \[D&]  \[E&]  \[F&]  \[G&]
% Sharp (Diesis):            \[A#]  \[B#]  \[C#]  \[D#]  \[E#]  \[F#]  \[G#]
% Minor:                     \[Am]  \[Bm]  \[Cm]  \[Dm]  \[Em]  \[Fm]  \[Gm]
% Flat and minor:            \[A&m] \[B&m] \[C&m] \[D&m] \[E&m] \[F&m] \[G&m]
% Sharp and minor:           \[A#m] \[B#m] \[C#m] \[D#m] \[E#m] \[F#m] \[G#m]

\endsong
