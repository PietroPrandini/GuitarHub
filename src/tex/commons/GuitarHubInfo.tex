\section*{Considerations}
This book is meant mainly for the guitar players who would like to have a chords book that is updated periodically without the loss of formatting and any other related problems.\par
It is highly recommended to print the content of this booklet and put it in a ring notebook - when a new song is added you can print it out and place it in a desired location. In order to avoid inconsistencies when adding the new songs to your collection, the pages are not labeled with any numbers. You can, for example, order the songs alphabetically.\par

\section*{Features}
\begin{itemize}
\item The booklet is available to be printed at once, including all the songs till date and then you can expand it by adding the newly released songs without the necessity to reprint the whole booklet;
\item the format of a single page is ISO A5 which supports the idea of  portability;
\item each song starts at a page with an even number to support the updating process;
\item creating songs with musical note names in both absolute and solfege systems;
\item easy and automatic transposition of the chords;
\item ability to add your favourite songs;
\item the book is licensed by a Free Culture License;
\item support for creating songs;
\item support for generating the booklets.
\end{itemize}

\section*{How to have a copy}
See \href{https://github.com/PietroPrandini/GuitarHub}{https://github.com/PietroPrandini/GuitarHub}

\begin{center}
  \qrcode[hyperlink,height=1.5cm]{https://github.com/PietroPrandini/GuitarHub}
\end{center}
