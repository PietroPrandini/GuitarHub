% Copyright 2018-2021 Pietro Prandini
%
% This file is part of GuitarHub.
%
% GuitarHub is free software: you can redistribute it and/or modify
% it under the terms of the GNU General Public License as published by
% the Free Software Foundation, either version 3 of the License, or
% (at your option) any later version.
%
% GuitarHub is distributed in the hope that it will be useful,
% but WITHOUT ANY WARRANTY; without even the implied warranty of
% MERCHANTABILITY or FITNESS FOR A PARTICULAR PURPOSE.  See the
% GNU General Public License for more details.
%
% You should have received a copy of the GNU General Public License
% along with GuitarHub.  If not, see <https://www.gnu.org/licenses/>.

\section*{Considerations}
This book is meant mainly for the guitar players who would like to have a chords book that is updated periodically without the loss of formatting and any other related problems.\par
It is highly recommended to print the content of this booklet and put it in a ring notebook - when a new song is added you can print it out and place it in a desired location. In order to avoid inconsistencies when adding the new songs to your collection, the pages are not labeled with any numbers. You can, for example, order the songs alphabetically.\par

\section*{Features}
\begin{itemize}
\item The booklet is available to be printed at once, including all the songs till date and then you can expand it by adding the newly released songs without the necessity to reprint the whole booklet;
\item the format of a single page is ISO A5 which supports the idea of  portability;
\item creating songs with musical note names in both absolute and solfege systems;
\item easy and automatic transposition of the chords;
\item ability to add your favourite songs;
\item the book is licensed by a Free Culture License;
\item support for creating songs;
\item support for generating the booklets.
\end{itemize}

\section*{How to have a copy}
See \href{https://github.com/PietroPrandini/GuitarHub}{https://github.com/PietroPrandini/GuitarHub}

\begin{center}
  \qrcode[hyperlink,height=1.5cm]{https://github.com/PietroPrandini/GuitarHub}
\end{center}
\newpage
