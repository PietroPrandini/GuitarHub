%\meter{4}{4}

%	\ifchorded
%	\beginverse* % * not count the verse
%		{\nolyrics Intro: }
%	\endverse
%	\fi

	\beginverse*\memorize % \memorize is used to set the chords you would like to use with ^ in the next verses
		%verse
		\[D]Mi son nato ne un paese strano
		ch'el se \[G]ciama \[A]Luga\[D]gnano,
		ghe la cesa senza campanìl ma le \[G]ore le \[A]bate \[D]stesso.
		I boceti a \[D7]sciàpi par la piassa nò gà \[G]idee da \[A]reali\[D]sàr.
		I vecioti i se cata a l'ostaria pa'n trì\[G]sete en \[A]compa\[D]gnia. \[B7]
	\endverse

	\beginverse*
		\[E]I palassi i \[E7]nasse come i fonghi \[A]dopo'n tempo\[B7]ràl de A\[E]gosto,
		i persegàri \[E7]i è sempre manco i mo\[A]ràri \[B7]no i ghè \[E]più. \[A7]
	\endverse

	\beginverse*
		\[D]E le strade i è taja i è sfàlta e \[G]dopo \[A]i è ri\[D]taja,
		è 'rivà \[D7]tanta gente noa, qualche \[G]vecio l'à ti\[A]rà 'l gam\[D]beto! \[D7]
	\endverse

	\beginverse*
		E mi che \[G]son 'n paesàn de \[F#m]sòca, vorea \[Em]dirve a tuti \[A]quanti:
		dovì essar \[Bm]fieri de stàr ne stò pa\[Em]ese, el pì \[E7]mato che ghe \[A]sìa,
		el pì ro\[D]gnoso el pì ciaco\[G]lòn de \[A]tuto el circon\[F#m]dario.
		Quel che se \[Bm]taja 'n diel ale Beca\[Em]rìe,
		a Manca\[A4]laqua l'è \[A7]morto dissan\[D]guà! \[B7]
	\endverse

	\beginverse*
		\[E]Na na na na \[E7]na na na na na na na \[A] na na \[B7]na na na \[E]na na
		\[E]Na na na na \[E7]na na na na na na na \[A] na na \[B7]na na na \[E]na
	\endverse

	\beginverse*
		Luga\[E]gnàn, el paese de la \[E7]cava Cà di Capri, de 'La Mela',
		de la fero\[A]vìa, che tanto 'n \[Am]pressia la nè portarà \[E]'ia
		come quatro \[B7]carte che sgola nel vento.
	\endverse

	\beginverse*
		Luga\[E]gnàn, el paese de le \[E7]sagre \echo{e son contento!}
		de le feste e de la bal\[A]doria,
		de ci fà i \[Am]schej come la giàra,
		de ci se \[E]grata la petara.
		Mondissie e no\[B7]tissie da osta\[E]ria. \[G7]
	\endverse

	\beginverse*
		\[C]Nemo tuti a \[C7]be'a tri quatro goti da \[F]'Milio o \[G7]dala Pal\[C]peta,
		\[C7]e vedarì che ogni rogna, ogni storia \[F]la vegna\[G7]rà ne\[C]gà.
		Tuto, tuto, \[C7]tuto se spianarà el \[F]paese el \[G7]cressa\[C]rà.
		Tuti, tuti \[C7]i sarà contenti a\[F]torno ala \[G7]stessa \[C]tòla, \[C7]
		novi e \[F]veci \[G7]resi\[C]denti, \[C7]
		novi e \[F]veci \[G7]resi\[C]denti. \[B7]
	\endverse

	\beginverse*
		\[E]Na na na na \[E7]na na na na na na na \[A] na na \[B7]na na na \[E]na na
		\[E]Na na na na \[E7]na na na na na na na \[A] na na \[B7]na na na \[E]na
	\endverse

%	\ifchorded
%	\beginverse* % * not count the verse
%		{\nolyrics Strum: }
%	\endverse
%	\fi

%	\textnote{} % Notes for both lyric and chorded songs
%	\musicnote{} % Notes visible only in chorded books (not visible in lyric mode)
%	\rep{n} % Repeat n times
%	\lrep ... \rrep \rep{n} % margins of the repeat

%	Writing chords
%
% Alphabetic note names:     A      B      C      D      E      F      G
% Solfedge note names:       LA     SI     DO     RE     MI     FA     SOL
%
%	Compatible notation:
%
% Naturals:                  \[A]   \[B]   \[C]   \[D]   \[E]   \[F]   \[G]
% Flat (Bemolle):            \[A&]  \[B&]  \[C&]  \[D&]  \[E&]  \[F&]  \[G&]
% Sharp (Diesis):            \[A#]  \[B#]  \[C#]  \[D#]  \[E#]  \[F#]  \[G#]
% Minor:                     \[Am]  \[Bm]  \[Cm]  \[Dm]  \[Em]  \[Fm]  \[Gm]
% Flat and minor:            \[A&m] \[B&m] \[C&m] \[D&m] \[E&m] \[F&m] \[G&m]
% Sharp and minor:           \[A#m] \[B#m] \[C#m] \[D#m] \[E#m] \[F#m] \[G#m]
