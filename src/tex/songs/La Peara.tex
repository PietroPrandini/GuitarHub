% Copyright 2018-2021 Pietro Prandini
% 
% This file is part of GuitarHub.
% 
% GuitarHub is free software: you can redistribute it and/or modify
% it under the terms of the GNU General Public License as published by
% the Free Software Foundation, either version 3 of the License, or
% (at your option) any later version.
% 
% GuitarHub is distributed in the hope that it will be useful,
% but WITHOUT ANY WARRANTY; without even the implied warranty of
% MERCHANTABILITY or FITNESS FOR A PARTICULAR PURPOSE.  See the
% GNU General Public License for more details.
% 
% You should have received a copy of the GNU General Public License
% along with GuitarHub.  If not, see <https://www.gnu.org/licenses/>.


%	\beginverse* % * not count the verse
%		{\nolyrics Intro: }
%	\endverse

	\beginverse\memorize % \memorize is used to set the chords you would like to use with ^ in the next verses
		%verse
		Noi\[D]altri semo \[A]quei che stà a Ve\[D]rona
		sen \[G]matti ma \[A]sen dei brai bu\[D]tei \[D7]
		e come la \[G]gente de altre cit\[A]tà
		ghemo la \[F#m]nostra speciali\[Bm]tà
		che \[Em]la se \[A]ciama pea\[D]rà.
	\endverse

	\beginchorus
	\textnote{Chorus A}
		La pea\[D]rà, la pearà, la pea\[F#m]rà
		de soli\[G]to la và ma\[A]gnà ala dumini\[F#m]ca
		con el \[G]lesso o col code\[A]ghin
		beendoghe \[F#m]drio en bicer de \[Bm]vin,
		la pea\[Em]rà le una tradis\[A]sion de la me cit\[D]tà.
	\endchorus

	\beginverse
		La na^ria fata con ^la miola de ^osso,
		ma ghe anca ^quei che no ^ghe le mete ^miga, ^
		basta en ^po de pan grat^tà,
		formaio, ^pear e brodo ^bon
		e wua^là ecco ^pronta la pea^rà.
	\endverse

	\beginchorus
	\textnote{Chorus B}
		La pea\[D]rà, la pearà, la pea\[F#m]rà
		de soli\[G]to la và ma\[A]gnà ala dumini\[F#m]ca
		le bona \[G]anca da pa\[A]rela,
		le bona \[F#m]anca pocià col \[Bm]pan,
		la pea\[Em]rà le bona \[A]anca riscal\[D]dà
	\endchorus

	\beginverse
		Te ^pol magnarla in^sieme coi ossi de ^porco
		o con en ^meso dele ^groste de for^maio, ^
		e ghe anca ^quei che cata su la ^tea
		se la ^fatto en po' de bru^sin,
		la pea^rà le bona ^anca en po' bru^sà.
	\endverse

	\textnote{Chorus A}

	\beginverse
		L'en ^gusto quando te te ^catti insieme a ma^gnarla,
		le un ^motivo per re^stare in compa^gnia, ^
		te ghe tiri ^su dele bele ^bale
		se i ghe ^mete dentro tanto ^pear,
		ma che bon^tà che ^le la pea^rà.
	\endverse

	\textnote{Chorus B}
	\textnote{Chorus A}
%	\textnote{} %for notes

%                 Do     Re     Mi     Fa     Sol    La     Si
%Naturali:        \[C]   \[D]   \[E]   \[F]   \[G]   \[A]   \[B]
%Bemolli:         \[C&]  \[D&]  \[E&]  \[F&]  \[G&]  \[A&]  \[B&]
%Diesis:          \[C#]  \[D#]  \[E#]  \[F#]  \[G#]  \[A#]  \[B#]
%Minori:          \[Cm]  \[Dm]  \[Em]  \[Fm]  \[Gm]  \[Am]  \[Bm]
%Bemolli minori:  \[C&m] \[D&m] \[E&m] \[F&m] \[G&m] \[A&m] \[B&m]
%Diesis minori:   \[C#m] \[D#m] \[E#m] \[F#m] \[G#m] \[A#m] \[B#m]
