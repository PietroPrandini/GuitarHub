% Copyright 2018-2021 Pietro Prandini
% 
% This file is part of GuitarHub.
% 
% GuitarHub is free software: you can redistribute it and/or modify
% it under the terms of the GNU General Public License as published by
% the Free Software Foundation, either version 3 of the License, or
% (at your option) any later version.
% 
% GuitarHub is distributed in the hope that it will be useful,
% but WITHOUT ANY WARRANTY; without even the implied warranty of
% MERCHANTABILITY or FITNESS FOR A PARTICULAR PURPOSE.  See the
% GNU General Public License for more details.
% 
% You should have received a copy of the GNU General Public License
% along with GuitarHub.  If not, see <https://www.gnu.org/licenses/>.

%	\beginverse* % * not count the verse
%		{\nolyrics Intro: }
%	\endverse

	\beginverse\memorize % \memorize is used to set the chords you would like to use with ^ in the next verses
		%verse
		\[G]Non c'è al mondo chi mi \[C]ami, \[G]
		non c'è stato mai ness\[C]uno \[G]
		in fondo alla mia \[D]vita, \[C]come Te.
		\[G]È con Te la mia par\[C]tita. \[G]
		Come sabbia fra le \[C]dita \[G]
		scorrono i miei \[D]giorni ins\[C]ieme a Te.
		\[F]Inquietudine, o malinconia:
		non c'è posto \[C]per loro \[G]in casa \[D]mia.
		\[F]Sempre nuovo è il tuo modo di
		inventare il \[C]gioco del \[G]tempo per \[Em]me.
	\endverse

	\beginchorus
		%chorus
		\[G]Nasc\[D]erà \[Em]dentro \[C]me,
		\[G]sul sil\[D]enzio che \[Em]abita \[C]qui,
		\[G]fior\[D]irà \[Em]un canto \[C]che
		\[G]mai ne\[D]ssuno ha can\[Em]tato per \[C]Te.
	\endchorus

	\beginverse
		^Se la strada si fa d^ura, ^
		come posso aver pa^ura? ^
		Nel buio della ^notte ^ci sei Tu.
		^Se mi assale la fat^ica ^
		di cancellare la sconf^itta, ^
		dietro ogni fe^rita ^sei ancora Tu.
		^È una cosa che non mi spiego mai:
		cosa ho tatto ^perché ^Tu scegliessi ^me?
		^Cosa mai dirò quando mi vedrai,
		quando dai conf^ini del ^mondo verr^ai?
	\endverse

%	\textnote{} % Notes for both lyric and chorded songs
%	\musicnote{} % Notes visible only in chorded books (not visible in lyric mode)
%	\rep{n} % Repeat n times

%	Writing chords
%
% Alphabetic note names:     A      B      C      D      E      F      G
% Solfedge note names:       LA     SI     DO     RE     MI     FA     SOL
%
%	Compatible notation:
%
% Naturals:                  \[A]   \[B]   \[C]   \[D]   \[E]   \[F]   \[G]
% Flat (Bemolle):            \[A&]  \[B&]  \[C&]  \[D&]  \[E&]  \[F&]  \[G&]
% Sharp (Diesis):            \[A#]  \[B#]  \[C#]  \[D#]  \[E#]  \[F#]  \[G#]
% Minor:                     \[Am]  \[Bm]  \[Cm]  \[Dm]  \[Em]  \[Fm]  \[Gm]
% Flat and minor:            \[A&m] \[B&m] \[C&m] \[D&m] \[E&m] \[F&m] \[G&m]
% Sharp and minor:           \[A#m] \[B#m] \[C#m] \[D#m] \[E#m] \[F#m] \[G#m]
