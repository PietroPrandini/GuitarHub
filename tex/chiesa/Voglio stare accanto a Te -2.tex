\beginsong{%Title
Voglio stare accanto a Te}[
%by={} % Authors, composers, and other contributors
%,cr={} % Copyright information
%,li={} % Licensing information
%,sr={} % Related scripture references
%,index={} % An extra index entry for a line of lyrics
%,ititle={} % An extra index entry for a hidden title
]

%\capo{0}
\transpose{10} % Automatic transpositions from +0 to +12 semitones

%	\beginverse* % * not count the verse
%		{\nolyrics Intro: }
%	\endverse

	\beginverse\memorize % \memorize is used to set the chords you would like to use with ^ in the next verses
		%verse
		\[G]Voglio stare \[Am]qui accanto a Te, \[G] \[C]
		per ado\[D]rare la tua pre\[C]sen\[G]za.
		\[Em] Io non posso vi\[Am]vere senza Te, \[G] \[C]
		voglio \[D]stare accanto a \[G]Te. \[C] \[G]
	\endverse

	\beginverse*
		^Voglio stare ^qui accanto a Te, ^ ^
		abi^tare la tua ^ca^sa,
		^ nel tuo luogo ^santo dimorar ^ ^
		per res^tare accanto a ^Te. ^ ^
	\endverse

	\beginchorus
		%chorus
		\[G]Accanto a te Si\[Am]gnore
		\[C]voglio dimo\[D]ra\[G]re, \[Em]
		gioire alla tua m\[Am]ensa,\[C] resp\[D]irando la tua \[Em]gloria.
		\[D] Del tuo \[C]amore io voglio \[G]viv\[Bm]ere \[Em]Signor,
		\[C] voglio \[D]star con Te,\[E&] voglio \[F]star con Te, \[G]Gesù.
	\endchorus

	\beginverse
		^Voglio stare ^qui accanto a Te, ^ ^
		per en^trare alla tua pre^sen^za.
		^ Io non posso vi^vere senza Te, ^ ^
		voglio ^stare accanto a ^Te. ^ ^
	\endverse
	
	\beginchorus
		\[D]Mio \[Em]Signor, Tu sei la \[C]mia for\[D]za,
		la \[Bm]gioia del mio \[Em]canto, la fort\[Am]ezza del mio \[C]cuor.
	\endchorus

	\beginverse
		^Voglio stare ^qui accanto a Te, ^ ^
		per ado^rare la Tua pre^sen^za.
		^ Nel tuo luogo ^santo dimorar ^ ^
		voglio ^star con Te, Ge^sù. ^ ^
	\endverse

	\beginchorus
		\[C] Voglio \[D]star con Te,\[E&] voglio \[F]star con Te, \[G]Gesù.
		\[C] Voglio \[D]star con Te,\[E&] voglio \[F]star con Te, \[G]Gesù.
	\endchorus

%	\textnote{} % Notes for both lyric and chorded songs
%	\musicnote{} % Notes visible only in chorded books (not visible in lyric mode)
%	\rep{n} % Repeat n times

%	Writing chords
%
% Alphabetic note names:     A      B      C      D      E      F      G
% Solfedge note names:       LA     SI     DO     RE     MI     FA     SOL
%
%	Compatible notation:
%
% Naturals:                  \[A]   \[B]   \[C]   \[D]   \[E]   \[F]   \[G]
% Flat (Bemolle):            \[A&]  \[B&]  \[C&]  \[D&]  \[E&]  \[F&]  \[G&]
% Sharp (Diesis):            \[A#]  \[B#]  \[C#]  \[D#]  \[E#]  \[F#]  \[G#]
% Minor:                     \[Am]  \[Bm]  \[Cm]  \[Dm]  \[Em]  \[Fm]  \[Gm]
% Flat and minor:            \[A&m] \[B&m] \[C&m] \[D&m] \[E&m] \[F&m] \[G&m]
% Sharp and minor:           \[A#m] \[B#m] \[C#m] \[D#m] \[E#m] \[F#m] \[G#m]

%	You can print the book with solfege note names
%	by uncomment a line of the GuitarChords.tex:
%	```
%	%  \notenamesin{A}{B}{C}{D}{E}{F}{G}
  \notenamesout{LA}{SI}{DO}{RE}{MI}{FA}{SOL}
 % Solfedge note names
%	```

\endsong
