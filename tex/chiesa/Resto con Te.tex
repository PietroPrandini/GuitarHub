\beginsong{%Title
Resto con Te}[
by={P.A. Sequeri, P. Stradi, M.T. Henderson, N.L. Uelmen, J.K. Belamide} % Authors, composers, and other contributors
%,cr={} % Copyright information
%,li={} % Licensing information
%,sr={} % Related scripture references
%,index={} % An extra index entry for a line of lyrics
%,ititle={} % An extra index entry for a hidden title
]

%\capo{0}
\transpose{0} % Automatic transpositions from +0 to +12 semitones

	\beginverse* % * not count the verse
		{\nolyrics Intro: | | \[G] | \[D4] | \[Em7] | \[C] |}
	\endverse

	\beginverse\memorize % \memorize is used to set the chords you would like to use with ^ in the next verses
		%verse
		\[C]Seme getta\[D]to nel mon\[Em]do,
		\[C]Figlio dona\[D]to alla ter\[G]ra,
		il Tuo sil\[C]enzio \[D] custodi\[Em]rò. \[D]
		\[C]In ciò che vi\[D]ve e che muo\[Em]re
		\[C]vedo il tuo vol\[D]to d’amo\[G]re,
		sei il mio Si\[C]gnore \[Am] e sei il mio \[D4]Dio. \[D]
	\endverse

	\beginchorus
		%chorus
		\[G]Io lo \[D]so che Tu \[Em]sfidi la mia mor\[C]te,
		\[G]io lo \[D]so che Tu \[Em]abiti il mio bu\[C]io.
		\[Am]Nell’at\[G]tesa del \[C]giorno che ver\[D4]rà\[D]
		Resto con | \[G]Te. |
	\endchorus

	\beginverse*
		{\nolyrics Strum: | \[D4] | \[Em7] | \[C] | \[G] | \[D4] | \[Em7] | \[C] |}
	\endverse

	\beginverse
		^Nube di man^dorlo in fio^re
		^dentro gli inver^ni del cuo^re
		è questo ^pane ^ che Tu ci ^dai. ^
		^Vena di cie^lo profon^do
		^dentro le not^ti del mon^do
		è questo ^vino ^ che Tu ci ^dai. ^
	\endverse

	\beginchorus
		%chorus
		\[G]Io lo \[D]so che Tu \[Em]sfidi la mia mor\[C]te,
		\[G]io lo \[D]so che Tu \[Em]abiti il mio bu\[C]io.
		\[Am]Nell’at\[G]tesa del \[C]giorno che ver\[D4]rà\[D]
		Resto con | \[C]Te. \[D] |
	\endchorus

	\beginverse*
		{\nolyrics Strum: | \[G] | \[C] \[D] | \[G] | \[Am] | \[Em] | \[Am] | \[D] |
			\[C] \[D] | \[Em] | \[C] \[D] | \[G] | \[C] | \[Am] | \[D4] | \[D] |}
	\endverse

	\beginchorus
		\[G]Tu sei \[D]Re di stel\[Em]late immensità\[C]
		\[G]e sei \[D]Tu il fu\[Em]turo che verrà\[C]
		\[Am]sei l’a\[G]more che \[C]muove ogni real\[D4]tà\[D]
		e Tu sei | \[G]qui | \[D] | \[C] |
		\[C] Resto con | \[G]Te. | \[D] | \[C] | \[C] | \rep{2}
		\[G]
	\endchorus

%	\textnote{} % Notes for both lyric and chorded songs
%	\musicnote{} % Notes visible only in chorded books (not visible in lyric mode)
%	\rep{n} % Repeat n times

%	Writing chords
%
% Alphabetic note names:     A      B      C      D      E      F      G
% Solfedge note names:       LA     SI     DO     RE     MI     FA     SOL
%
%	Compatible notation:
%
% Naturals:                  \[A]   \[B]   \[C]   \[D]   \[E]   \[F]   \[G]
% Flat (Bemolle):            \[A&]  \[B&]  \[C&]  \[D&]  \[E&]  \[F&]  \[G&]
% Sharp (Diesis):            \[A#]  \[B#]  \[C#]  \[D#]  \[E#]  \[F#]  \[G#]
% Minor:                     \[Am]  \[Bm]  \[Cm]  \[Dm]  \[Em]  \[Fm]  \[Gm]
% Flat and minor:            \[A&m] \[B&m] \[C&m] \[D&m] \[E&m] \[F&m] \[G&m]
% Sharp and minor:           \[A#m] \[B#m] \[C#m] \[D#m] \[E#m] \[F#m] \[G#m]

%	You can print the book with solfege note names
%	by uncomment a line of the GuitarChords.tex:
%	```
%	%  \notenamesin{A}{B}{C}{D}{E}{F}{G}
  \notenamesout{LA}{SI}{DO}{RE}{MI}{FA}{SOL}
 % Solfedge note names
%	```

\endsong
